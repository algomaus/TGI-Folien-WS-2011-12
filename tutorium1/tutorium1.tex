\documentclass{beamer}
\usetheme[deutsch]{KIT}

\KITfoot{foo bar}%TODO
\usepackage[utf8]{inputenc}
\usenavigationsymbols

\title{Theoretische Grundlagen der Informatik}
\subtitle{Tutorium}
%TODO Authors not displayed :-S 
\author{Moritz von Looz, Simon Stroh}

\institute[ITI]{Intitute für Theoretische Informatik}

\TitleImage[height=\titleimageht]{../pics/tmaschine.png}

\begin{document}

\begin{frame}
  \maketitle
\end{frame}

\begin{frame}
  \frametitle{Übersicht}
  \tableofcontents
\end{frame}

\section{Organisatorisches}
\begin{frame}
	\frametitle{Organisatorisches - Zum Übungsbetrieb}
	\begin{itemize}
		\item \textbf{Abgabe:} \emph{Handschriftlich} in Zweiergruppen.
		\item \textbf{Schein:} 
		\begin{itemize}
			\item Klausurbonus
			\item Wahrscheinlich ein Notenschritt
			\item Ab 50\% der erreichbaren Punkte
		\end{itemize}
		\item Tutoriumsmaterial und Aktueller Punktestand online.
		\begin{itemize}
			\item URL wird noch bekanntgegeben
			\item Wer Punkte online einsehen will, email in Liste eintragen
		\end{itemize}
	\end{itemize}
\end{frame}
\begin{frame}
	\frametitle{Organisatorisches - Zum Tutorium}
	\begin{itemize}
		\item Stoff soll wiederholt werdet
		\item Dabei Fokus auf Übungsbetrieb
		\item Fragen/Vorschläge/Anmerkungen wilkommen!
		%\item Wer Tee möchte darf sich bedienen :)
	\end{itemize}
\end{frame}
\section{Endliche Automaten und reguläre Sprachen}
\subsection{Formale Sprachen}
\begin{frame}
	\frametitle{Kurze Wiederholung: Formale Sprachen}
	Eine \emph{Formale Sprache} $L$ ist eine Teilmenge aller Wörtern über einem endlichen Alphabet $\Sigma$. Also $L \subseteq \Sigma^*$\\[0.3cm]
	Beispiele:
	\KITframe[yes the background would be nice in gray, thanks!]
		{\parbox{\textwidth}{\begin{itemize}
			\item $\Sigma = \{ 0, 1 \}, L = \{w11z\,|\,w,z \in \Sigma^*\}$
			\begin{itemize}
			\item Die Menge aller Wörter die ''11'' enthalten.
			\end{itemize}
		\end{itemize}}
	}\\[0.2cm]
Im Allgemeinen kann man Formale Sprachen sehr frei Angeben: 
	\KITframe[Yeah, I think I'll stick to gray as a background color. Thanks again!]
		{\parbox{\textwidth}{\begin{itemize}
			\item $\Sigma = \{ 0, 1 \}, L = \{w|\,w \in \Sigma^*, w \mbox{ hat eine gerade Anzahl an $1$en)}\}$
			\begin{itemize}
			\item Die Menge aller Wörter die eine gerade Anzahl an Einsen enthalten.
			\end{itemize}
		\end{itemize}}
	}
\end{frame}
\subsection{Reguläre Ausdrücke}
%TODO Frame to Reg Langs
%TODO Frame to Reg Exes
\subsection{Deterministische endliche Automaten}
\begin{frame}
\frametitle{Deterministische Endliche Automaten}
        Ein deterministischer endlicher Automat $M$ ist ein 5-Tupel
        \[
        M= (Q,\Sigma,\delta,S,F).
        \]
        \begin{itemize}
        \item $Q$:  endliche Zustandsmenge
        \item $\Sigma$:    endliches Alphabet
        \item $\delta$:   Zustandsübergangsfunktion $Q\times \Sigma \rightarrow Q$
        \item $S$:   Startzustand $\in Q$
        \item $F$:   Endzustandsmenge $\subseteq Q$
        \end{itemize}
\end{frame}
\begin{frame}
	\frametitle{DEA: Beispiel}%TODO
\end{frame}
\begin{frame}
	\frametitle{DEA: Aufgaben}
	\begin{enumerate}
	\item 
		Lösen Sie folgendes Rätsel mit Hilfe eines deterministischen
		endlichen Automaten:
		\begin{quote}
		  Es stehen drei Wasserkrüge mit einem Fassungsvermögen
		  von 3, 5 bzw. 7 $l$ zur Verfügung, um eine Wassermenge von
		  einem Liter abzumessen, d.~h. in einem der Krüge soll sich genau
		  diese Menge Wassers befinden. Zu Beginn sind der kleinste und
		  der größte Krug gefüllt. Da Ihr Augenmaß schlecht
		  ist, darf Wasser nur so von einem Krug in einen anderen
		  gegossen werden, dass der eine ganz geleert oder der andere
		  ganz gefüllt wird (ohne dass Wasser verschüttet wird).
		\end{quote}
		Geben Sie den Übergangsgraphen eines Automaten an, dessen
		akzeptierte Sprache genau die zulässigen lösenden
		Umfüllreihenfolgen kodiert, sowie ein kürzestes Lösungswort.
	\end{enumerate}
\end{frame}
\begin{frame}
	\frametitle{DEA: Lösung}
	\begin{enumerate}
	\item TODO %TODO?
	\end{enumerate}
\end{frame}
\begin{frame}
	\begin{enumerate}
	\setcounter{enumi}{1}
	\item Aufgabe: Konstruiere einen DEA der alle durch 5 teilbaren Zahlen akzeptiert. Als Eingabe erhält der Automat dabei die Zahl in ihrer binären Darstellung. Also ist $\Sigma = \{0, 1\}$. Z.B. soll Automat $10_{10} = 1010_{2}$ akzeptieren, aber $7_{10} = 111_{2}$ ablehnen.
	\pause \\[10pt]
	Tip: Restklassen als Zustände modellieren
	\end{enumerate}
\end{frame}
\begin{frame}
	\frametitle{DEA: Lösung}
	\begin{enumerate}
	\setcounter{enumi}{1}
	\item Idee
		\begin{itemize}
		\item $Q = (q_0, q_1, q_2, q_3, q_4)$\\
		\item $\Sigma = \{0,1\}$
		\item $\delta(q_n, c) \rightarrow q_{n \cdot 2 + c\mbox{ mod }5}$ mit $c\in\Sigma$\\
		\item $S = q_0$\\
		\item $F = \{q_0\}$
		\end{itemize}
	\end{enumerate}
\end{frame}
\begin{frame}
\frametitle{DEA: Aufgabe}
\begin{enumerate}
\setcounter{enumi}{2}
\item Geben Sie einen regulären Ausdruck für die vom DEA mit nachfolgendem Zustandsgraphen erkannte Sprache an:
\begin{center}\includegraphics[scale=0.7]{pics/dea1.pdf}\end{center}
\end{enumerate}
\end{frame}
\begin{frame}
	\frametitle{DEA: Lösung}
	\begin{enumerate}
	\setcounter{enumi}{2}
	\item TODO %TODO
	\end{enumerate}
\end{frame}
\subsection{Nichtdeterministische endliche Automaten}
\begin{frame}
\frametitle{Nichtdeterministische Endliche Automaten}
        Ein nichtdeterministischer endlicher Automat $M$ ist ein 5-Tupel
        \[
        M= (Q,\Sigma,\delta,S,F).
        \]
        \begin{itemize}
        \item $Q$:  endliche Zustandsmenge
        \item $\Sigma$:    endliches Alphabet
        \item \textcolor{red}{$\delta$:   Zustandsübergangsfunktion $Q\times (\Sigma \cup \varepsilon) \rightarrow 2^Q$}
        \item $S$:   Startzustand $\in Q$
        \item $F$:   Endzustandsmenge $\subseteq Q$
        \end{itemize}
\end{frame}
\begin{frame}
	\frametitle{NEA: Beispiel}%TODO
\end{frame}
\begin{frame}
	\frametitle{NEA: Aufgabe}
 Welche Sprache akzeptiert der nichtdeterministische endliche Automat
 zu dem folgenden Zustandsgraphen?
 \begin{center}
   \vspace{-6ex}
   \includegraphics[scale=.8]{pics/NEA_01}
 \end{center}
\end{frame}
\begin{frame}
	\frametitle{NEA: Lösung}
	$$L = c(b^* \cup a^*c \cup cc^*)$$
 \begin{center}
   \vspace{-6ex}
   \includegraphics[scale=.8]{pics/NEA_01}
 \end{center}
	
\end{frame}
\begin{frame}
	\frametitle{Äquivalenz von NEA und DEA}
	Wir wissen aus der VL: NEA und DEA sind äquivalet.
	Wie wandelt man nun einen NEA in einen DEA um?
\end{frame}
\subsubsection{Schritt 1: Eliminieren von Epsilon-Übergängen}
\begin{frame}
	\frametitle{Schritt 1: Eliminierung von $\varepsilon$-Übergängen}
	%TODO
\end{frame}
\subsubsection{Schritt 2: Potenzmengenkonstruktion}
\begin{frame}
	\frametitle{Schritt 2: Potenzmengenkonstruktion}
	%TODO
\end{frame}
\subsection{Konstruktion eines DEA aus einem NEA}
%TODO: hier muss noch formatierung + Lösung rein
\begin{frame}
  Über dem Alphabet $\Sigma = \{a,b\}$ sei der reguläre
  Ausdruck
  $r := {(a \cup (ab (b)^* ba))^*}$
  gegeben.
  \begin{enumerate}
    \setlength{\itemsep}{0ex}
  \item Geben Sie einen NEA an, der $L(r)$ erkennt. Begründen Sie
    kurz die Korrektheit Ihres Automaten, ein formaler
    Korrektheitsbeweis ist jedoch nicht erforderlich.\\
    (Hinweis: Es gibt einen NEA mit 3 Zuständen.)
  \item Konstruieren Sie zu dem von Ihnen angegebenen NEA einen
    äquivalenten DEA mittels Potenzmengenkonstruktion.
  \end{enumerate}
\end{frame}
\begin{frame}
Gegeben sei der NEA ${\cal A}=(\{s,q,f\},\{a,b,c\},\delta,s,\{f\})$, wobei
die Übergangsfunktion $\delta$ gegeben ist durch:
\begin{center}
$\begin{array}{r|cccc}
&\varepsilon & a & b & c\\\hline
s & \{q,f\} & \emptyset & \{q\} &\{f\}\\
q &  \emptyset & \{s\} & \{f\} & \{s,q\}\\
t & \emptyset & \emptyset &  \emptyset &  \emptyset\\
\end{array}$
\end{center}
\begin{enumerate}
\item Geben Sie zu dem Automaten ${\cal A}$ den Übergangsgraphen an und eliminieren
Sie die $\varepsilon$-Übergänge.
\item Ermitteln Sie mittels Potenzmengenkonstruktion den zu ${\cal A}$ äquivalenten
DEA. Geben Sie hierbei die Übergangsfunktion tabellarisch an.
\end{enumerate}

\end{frame}

\end{document}
