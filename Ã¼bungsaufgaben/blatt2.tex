\documentclass{article}
\usepackage[utf8]{inputenc}
\usepackage{tikz}
\usepackage{amsmath}
\usepackage{ifthen}
\usepackage[top=4cm,bottom=3cm,left=3cm,right=3cm]{geometry}
\usetikzlibrary{automata}
\newboolean{sol}
\newboolean{ex}


%%%% Config here

% Show excercises
\setboolean{ex}{true}
% Show solutions
\setboolean{sol}{true}
%%%

\usepackage{float}
\newcommand{\aufgaben}[2]{
\ifthenelse{\boolean{ex}}{
\ifthenelse{\boolean{sol}}{
	\subsection{Aufgaben}
}{}}{}
\ifthenelse{\boolean{ex}}{#1}{}
\ifthenelse{\boolean{ex}}{
\ifthenelse{\boolean{sol}}{
	\subsection{Lösungen}
}{}}{}
\ifthenelse{\boolean{sol}}{#2}{}
}

\begin{document}
\ifthenelse{\boolean{ex}}{
	\ifthenelse{\boolean{sol}}{
		\title{Übungsaufgaben 2 mit Lösung}
	}{
		\title{Übungsaufgaben 2}
	}
}{
	\ifthenelse{\boolean{sol}}{
		\title{Lösungen 2}
	}{
		\title{Leeres Blatt 2}
	}
}
\author{Simon Stroh und Moritz von Looz}
\maketitle
\section{Ch-2 Sprachen} 
\aufgaben{.
\begin{enumerate}
\item Zeige das die Menge der Sprachen aus CH-2 unter Vereinigung abgeschlossen sind.
\item Zeige unter Verwendung der Sprachen $\{a^mb^nc^n \mid  m,n > 0\}$ und $\{a^nb^nc^m \mid  m,n > 0\}$, dass die Menge die kontextfreien Sprachen nicht unter Schnittbildung abgeschlossen ist.
\item Zeige mit 2. und DeMorgan, dass die kontextfreien Sprachen nicht unter Komplementbildung abgeschlossen sind.
\item Warum funktioniert folgender ``Beweis'' nicht:
\begin{quote}
Die Menge der Kontextfreien Sprachen ist unter dem Keeleschen-Stern Operator abgeschlossen. Für eine Grammatik $G$ mit Startsymbol $S$ welche die Sprache $A$ generiert, füge die Regel $S \rightarrow SS$ ein, damit generiert $G'$ jetzt $A^*$
\end{quote}
\end{enumerate}
}{.
\begin{enumerate}
\item Seien $G$, $H$ Grammitken für Sprachen aus CH-2 mit Startsymbolen $S_G$, $S_H$. Konstruiere Grammatik $G'$ mit Startsymbol $S$ und den Regeln aus $G$, $H$ sowie der Regel $S \rightarrow S_G \mid  S_H$.
\item Beide Sprachen sind offensichtlich kontextfrei (eine Grammatik ist leicht anzugeben). Der Schnitt ist die bekanntermaßen nicht kontextfreie Grammatik $\{a^nb^nc^n\mid  n > 0\}$
\item Seien $L_1$, $L_2$ kontextfrei. Angenommen die kontextfreien Sprachen wären unter Komplementbildun abgeschlossen. Dann wäre $\overline{L_1} \cup \overline{L_2}$ und eben auch $\overline{\overline{L_1} \cup \overline{L_2}} = L_1 \cap L_2$ kontextfrei
\item Betrachte etwa die Grammatik $G$ (für Sprache $A$) gegeben durch $S \rightarrow \# \mid  aSb$. Das nach Anleitung konstruierte $G'$ mit $S \rightarrow \# \mid  aSb \mid  SS$ kann das Wort $a\#\#b$ ablieten, welches nicht in $A*$ liegt.
\end{enumerate}
}
\section{Chomsky Normalform / CYK}
\aufgaben{.
\begin{enumerate}
\item Gib eine Grammatik an, die die Sprache $L_1 = \{a^ib^jc^k \mid  i = j oder j = k$ erzeugt
\item Gib eine Grammatik in Chomsky Normalform an, die die Sprache $L_2 = \{a^nb^n \mid  n \geq 1\}$ erzeugt
\end{enumerate}
}{.
\begin{enumerate}
\item Konstruiere die Vereinigung wie in Aufgabe 1.1 für die Sprachen aus Aufgabe 1.2
\item Die Grammatik sei durch die Produktionen gegeben:
\begin{align*}
S&\rightarrow AB\mid AX\\
X&\rightarrow SB\\
A&\rightarrow a \\
B&\rightarrow b \\
\end{align*}
\end{enumerate}
}
\section{Ogdens und Pumping Lemma}
\aufgaben{.
Eine Aufgabe
}{.
Eine Lösung
}
\section{Greibach Normalform und Kellerautomaten}
\aufgaben{.
Eine Aufgabe
}{.
Eine Lösung
}

\end{document}
